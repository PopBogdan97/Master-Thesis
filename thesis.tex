% Document template suitable for use as a LaTeX master-file for master's
% thesis in University of Turku Department of Computing.

% HOW TO USE? See https://ttweb.utugit.fi/thesis/doc/overview/

\documentclass[language=english,version=final,mainfont=none,sharelatex=false]{utuftthesis}
\setcounter{secnumdepth}{2}
\setcounter{tocdepth}{2}
\usepackage{float}
\usepackage[caption=false]{subfig}

% Define the algorithm environment
%\makeatletter
\providecommand\textquotedblplain{%
  \bgroup\addfontfeatures{Mapping=}\char34\egroup}
\providecommand{\tabularnewline}{\\}
\floatstyle{ruled}
\newfloat{algorithm}{tbp}{loa}
\providecommand{\algorithmname}{Algoritmi}
\floatname{algorithm}{\protect\algorithmname}
%\makeatother

\addbibresource{references.bib}

\begin{document}

\pubyear{2021}
\pubmonth{10}
\publab{Huawei}
\publaben{Huawei - HERS}
\pubtype{gradu}
\title{Migration of Secure Enclaves}
\author{Vasile Adrian Bogdan Pop}
\supervisors{Arto Niemi, "Seppo Virtanen"}

\maketitle
\chapter*{Abstract} % senza numerazione
\label{abstract}

\addcontentsline{toc}{chapter}{Abstract} % da aggiungere comunque all'indice







% mandatory
\tableofcontents

% if you want a list of figures
\listoffigures

% if you want a list of tables
\listoftables

% if you want a list of acronyms
\listofacronyms

% change the name if the default doesn't sound right
\renewcommand{\algorithmname}{\listingscaption}

% The thesis starts here.

\begin{comment}
To better organize things, create a new tex hellllo file for each chapter
and input it below.

Avoid using the å, ä, ö or <space> characters in referred names and
underscores \_ in file names (may break hyperref).

Good luck!
\end{comment}

\chapter{Introduction}
\label{cha:introduction}


\section{Sec 1}
\label{sec:secone}

\chapter{Background}
\label{cha:background}


\section{Security Principles}

\section{Confidential Computing}

\section{TCB - Trusted Computing Base}

\section{TEE - Truseted Execution Environment}

\section{Remote attestation}



% Other concepts useful to understand the Secure Enclaves
\chapter{Secure Enclaves}
\label{cha:secenc}


\section{State of the Art}

\section{Existing implementation details}

\section{Existing problems with the actual implementations}

\chapter{Secure Enclave Migration}
\label{cha:secencmig}


\section{Sec 1}
\label{sec:secone4}

\chapter{Implementation}
\label{cha:implementation}


\section{Sec 1}
\label{sec:secone5}

\chapter{Conclusions}
\label{cha:conclusion}



%\input{file_name_of_chapter_x}
%\input{file_name_of_chapter_y}

% The thesis main content ends here.

\printbibliography

\begin{comment}
Important! Create the appendix chapters with command \textbackslash appchapter\{some
name\} instead of \textbackslash chapter\{some name\} for the automagic
page counting to work!
\end{comment}


% \appchapter{Liitedokumentti}
% 
% Liitteen ohjelmakoodi \ref{alg:Tyyppiluokka-Monad} kuvaa matemaattisen
% monadirakenteen pohjalta rakentuvan Haskellin tyyppiluokan. Tyyppiluokan
% voi nähdä eräänlaisena abstraktina ohjelmointirajapintana (API\nomenclature[API]{API}{Application Programming Interface}),
% joka muodostaa ohjelmoijalle abstraktin ohjelmointikielen käyttöliittymän
% (UI\nomenclature[UI]{UI}{User Interface}).
% 
% \begin{algorithm}[tbh]
% \begin{minted}{haskell}
% class Monad m where
%     ( >>= )         :: m a -> (a -> m b) -> m b
%     return          :: a                 -> m a
% 
%     fail            :: String            -> m a
%     (>>)            :: m a -> m b        -> m b
%     m >> k          =  m >>= \_ -> k       -- default
% 
% instance Monad IO where  ...               -- omitted
% \end{minted}
% 
% \caption{Tyyppiluokka 'Monad'.\label{alg:Tyyppiluokka-Monad}}
% \end{algorithm}
% 
% \newpage{}
% 
% Ensimmäisen liitteen toinen sivu. Ohjelmalistaus \ref{alg:Monadin-kayttoa}
% demonstroi vielä monadin käyttöä.
% 
% \begin{algorithm}[tbh]
% \begin{minted}{haskell}
% main =
% return "Your name:" >>=
% putStr >>=
% \_ -> getLine >>=
% \n -> putStrLn ("Hey " ++ n)
% \end{minted}
% 
% \caption{Monadin käyttöä.\label{alg:Monadin-kayttoa}}
% \end{algorithm}
% 
% 
% \appchapter{Liitedokumentti 2}
% 
% Tässä esimerkki\pagebreak{}
% 
% toisesta kaksisivuisesta liitteestä.
\end{document}
